\documentclass[a4paper,11pt]{article}

\usepackage{revy}
\usepackage[utf8]{inputenc}
\usepackage[T1]{fontenc}
\usepackage[danish]{babel}

\revyname{Biorevy}
\revyyear{2022}
\version{0.1}
\eta{$2:20$ minutter}
\status{Færdigærdig}
\responsible{Nina}

\title{Wonderful World}
\author{Nina}
\melody{Luis Armstrong: ``What a Wonderful World''}

\begin{document}
\maketitle

\begin{roles}
    
\role{S1}[Nina]
\role{S2}[Freja]
\end{roles}

\begin{props}
    \prop{Måske affald og en hval}[Person, der skaffer]
\end{props}

\textbf{Bemærkninger til Teknikken:} Lys ned på knæfald til sidst \\Måske en AV video der passer til teksten

\begin{song}
  \sings{S} Jeg ser blomsterne \hspace{0.5cm} på en rapsmark  
Jeg ser et træ   \hspace{0.5cm}  i en by-park
Og så undrer jeg mig
Hvor er naturen hen’?

 \sings{S} Jeg ser mikroplast  \hspace{0.5cm}   og husaffald
Det svømmer rundt  \hspace{0.5cm}   samm'n med en hval
Og så undrer jeg mig
Hvorn' har náturen det?

\sings{S} De mange insekter  \hspace{0.5cm} på forruden er væk
De købte klimakvoter  \hspace{0.5cm}   er en kat i en sæk
Snart er grænserne nået \hspace{0.5cm}  for hvad kloden kan tål’
Drivhuseffekten \hspace{0.5cm}   gør jorden til et bål

\sings{S}Og en dag får jeg  \hspace{0.5cm}   børn, som vil se
Hvad vi har gjort   \hspace{0.5cm}   åh ak åh ve
Og så spørger de mig
Hvorfor gjorde I ik’ no’et?
Og så græmmer jeg mig
Natúren må se sig slået

\end{song}

\end{document}

