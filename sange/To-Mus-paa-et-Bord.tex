\documentclass[a4paper,11pt]{article}

\usepackage{revy}
\usepackage[utf8]{inputenc}
\usepackage[T1]{fontenc}
\usepackage[danish]{babel}

\revyname{Biorevy}
\revyyear{2022}
\version{0.1}
\eta{$3-4$ minutter}
\status{Færdig}
\responsible{Nina}

\title{To mus på et bord}
\author{Lukas, Søren, Freja og Anna}
\melody{Otto Brandenburg: ``To Lys på et Bordl''}

\begin{document}
\maketitle

\begin{roles}
    \role{S}[Nina] 
\end{roles}

\begin{props}
    \prop{Skalpel}[Person, der skaffer]
\prop{To tøjmus}[Person, der skaffer]
\prop{Et bord}[Person, der skaffer]
\prop{Hjerte}[Person, der skaffer]

\end{props}

\textbf{Bemærkninger til Bandet:} Er skrevet én til én på originalen, men måske det lange mellemstykke uden sang skal forkortes eller fjernes. Spørg Nina om dette.

\begin{song}

  \sings{S}[omkv]To mus, på et bord
Tre små rus-biologer
Fire glas, fyldt med sprit, der står klar
Fem små snit, med skalpel
Seks små skrig, tre fra hver
Og vi så deres hjerteblod

  \sings{S} Kun din fantasi
Sætter grænser for det
Der kan ske, her til OD i dag
For de liv, som vi ta’r
Der på sjælen gi’r ar
Har vi glemt, når vi går herfra

  \sings{S} Hvordan er det sket
Har vi slet ikke set
At det snart er slut med OD
Og kurset der kom
Blev så håbløs og tomt
Jeg vil bare ha’ zoologi

  \sings{S} [omkv] To mus, på et bord
Tre små rus-biologer
Fire glas, fyldt med sprit, der står klar
Fem små snit, med skalpel
Seks små skrig, tre fra hver
Og vi så deres hjerteblod

  \sings{S} Næste uge vi skal
Splitte tre frøer ad
Og en torsk skal have hjernen ud
Og snittet bli'r lagt
Så fint dorsoventralt
I det levende søpindsvin

  \sings{S}[omkv]To mus, på et bord
Tre små rus-biologer
Fire glas, fyldt med sprit, der står klar
Fem små snit, med skalpel
Seks små skrig, tre fra hver
Og vi så deres hjerteblod

\end{song}

\end{document}

