\documentclass[a4paper,11pt]{article}

\usepackage{revy}
\usepackage[utf8]{inputenc}
\usepackage[T1]{fontenc}
\usepackage[danish]{babel}

\revyname{Biorevy}
\revyyear{2022}
% HUSK AT OPDATERE VERSIONSNUMMER
\version{0.1}
\eta{$2$ minutter}
\status{Færdig}
\responsible{En revyst}

\title{Blodrusturen}
\author{Søren \& Tristan}

\begin{document}
\maketitle

\begin{roles}
   \role{M1}[Nethe] Myg 1, kvindelig blodrus
   \role{M2}[Oscar] Mandlig blodrus (dansemyg)
   \role{M3}[Ania] Blodrusvejleder
\end{roles}

\begin{props}
    \prop{1 stor kanyle}[-]
    \prop{1 spytspand}[-]
    \prop{3 store sugerør}
    \prop{“Vin”, noget der skal ligne vin/blod}[-]
    \prop{1 bord}[-]
    \prop{“Rushat” med antenner}
    \prop{Myggevinger}
    \prop{Evt. stor fluesmækker til at nakke M2}
\end{props}

\begin{sketch}
\scene{lys op}

\scene{M1, M2 og M3 starter på scenen. M1 og M2 er rusmyg og M2 er tydeligvis en han} 

\says{M1} Uh, jeg har sådan glædet mig! Hvor er det fedt at være med på blodrustur! Hvad skal vi lave?

\says{M3} Jo, nu skal i se. Vi har jo alle forberedt noget hjemmefra. Og siden vi har fået pålagt flere diversitet krav er i år jo første gang vi har taget en han med på biologi blodrusturen. Velkommen til. Lad os starte med hvad jeg har taget med..

\says{AV} summelyd

\scene{M3 går ud bag bagtæppet og kommer ind igen med en kanyle med blod i}

\says{M1} Ej hvor spændende, det her smager slet ikke som nektar!

\says{M3} Den her er en meget klassisk og velsmagende og rund buket direkte fra arterien. Godt iltet og en god basis for den videre smagning.

\says{M1} Ja det kan jeg godt se, fornemmer jeg lidt frugtige undertoner?

\says{M3} Ja natløbet er lige slut så den har ligget på kirse de sidste fire timer. Læg også mærke til de flotte gardiner. Skal vi tage din? (peger på M1)

\scene{M1 går om bag scenen, M2 er blevet tydeligt mere beruset}

\says{M3} Spændende, svagt mousserende med lidt blålige toner. Venøst, hvis jeg ikke helt tager fejl?

\says{M1} Lige præcis, jeg tænkte det kunne være sjovt med lidt kontrast! Og alting er jo sjovere med bobler.

\says{M3} Hva.. Jeg synes stadig at jeg fornemmer lidt tung frugt i den her, og den dufter meget ens. Nu spørger jeg måske dumt, men er det fra samme farm som den forrige?

\says{M1} Hehe, ja du har opdaget det! Jeg kunne simpelthen ikke holde mig fra dem, den duftede bare så himmelsk! Men for ikke bare at gå i dine fodspor, så udvalgte jeg nøje mit terroir fra en mere sydligt liggende region. Mere præcist, den højre storetå.

\says{M3} Fornemt, lad os gå flux videre til den sidste

\says{M2} Æhhhh jo, min li “hik” ger li’ deer

\says{M3}  Ah ja jeg fornemmer du også er glad for vores favorit farm?

\scene{M2 går om bagved, han jamrer}

\says{M2} Hvordan faaanden virker den her, hvor åbner man? Er det her… ahhh

\says{AV} Klaskelyd

\scene{lys op}
\scene{De går ud} = regibemærkning
\scene{lys ned}
\end{sketch}

\end{document}
