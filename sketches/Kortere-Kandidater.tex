\documentclass[a4paper,11pt]{article}

\usepackage{revy}
\usepackage[utf8]{inputenc}
\usepackage[T1]{fontenc}
\usepackage[danish]{babel}

\revyname{Biorevy}
\revyyear{2022}
% HUSK AT OPDATERE VERSIONSNUMMER
\version{0.1}
\eta{$2:30$ minutter}
\status{færdig}
\responsible{Isolde}

\title{Kortere Kandidater}
\author{Lukas, Søren og Tristan}

\begin{document}
\maketitle

\begin{roles}
    \role{R}[Ania] Rektor Henrik
\role{D}[Isolde] Prorektor David  
\role{K}[Freja] Prorektor Kristian
\role{J}[Oscar] Universitets Direktør Jesper  
\end{roles}

\textbf{Bemærkninger til Teknikken:} Lys som når man slikker på sandpapir
%% Ikke gyldig
% \\Møde lyde - papir knitren

\begin{sketch}

\scene{R, D, K og J og står på scenen}
\scene{Sort lys ned}

\says{R} Jeg er Henrik Caspar Wegener og vi er blevet pålagt at forkorte vores kandidater her på KU. Nogle forslag til hvordan vi håndterer dette, her oppe fra direktionsgangen? 

\says{K} Jeg er prorektor for uddannelse Kristian Cedervall Lauta og jeg tænker det mest logiske må være et standard højdekrav. 170 cm lyder rimeligt? Så sparer vi også penge på de nye døre

\says{R} Jeg tror det vil rykke lidt for meget ved kønsbalancen…

\says{K} Så måske skal vi holde det krav til DIKU?

\says{R, K, J og D} Vedtaget!

\says{R} Men det er kun en uddannelse, vi mangler stadig at forkorte en del flere?

\says{J} Jeg er universitetsdirektør Jesper Olesen og jeg tænker vi måske kan kigge lidt mere overordnet. Der er nemlig gået fuldstændig inflation i længden af kandidattitler!

\says{D} Jeg, Prorektor David Dreyer Lassen, synes det er en god idé! Så kan vi kalde det fødeskab i stedet for fødevare videnskab!

\says{J} og Splat-Kult i stedet for Spansk og Latinamerikansk Sprog og Kulturanalyse!

\says{D} Anvendt Kulturanalyse er nu bare Ande-Kult!

\says{R} Joo mon de vil godtage den løsning oppe i ministeriet. Hvad med dig Prorektor for Forskning har du en idé? Du har været så stille her til dagens møde

\says{K} Jo Hr. Rektor men det er jo bare fordi vi egentlig mangler at snakke om det der med de etårige kandidater…

\says{R} etårige kandidater. Det har potentiale. Så mange penge sparet!

\says{K} Ja lige præcis!

\says{R} Når folk går direkte fra fødegangen til specialet vil vores samfund spare så mange penge!

\says{K} Ja etårige kandidater bliver til treårige Phd’er…

\says{R} De studerende vil få PostDoc stillinger før deres syv års fødselsdag! De penge vi vil spare på at nedlægge alle bachelor uddannelser, gymnasier, grundskoler og børnehaver

\says{R og K} SÅ MANGE FORSKNINGSMIDLER!



\scene{Sort lys op.}

\end{sketch}
\end{document}
