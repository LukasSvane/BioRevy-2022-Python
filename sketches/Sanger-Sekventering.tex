\documentclass[a4paper,11pt]{article}

\usepackage{revy}
\usepackage[utf8]{inputenc}
\usepackage[T1]{fontenc}
\usepackage[danish]{babel}

\revyname{Biorevy}
\revyyear{2022}
% HUSK AT OPDATERE VERSIONSNUMMER
\version{0.1}
\eta{$2$ minutter}
\status{Færdig}
\responsible{En revyst}

\title{Sanger-Sekventering}
\author{Lukas og Freja}

\begin{document}
\maketitle

\begin{roles}
   \role{a}[Laura] sanger a
\role{b}[Line] sanger b
\role{c}[Andrea] sanger c
\role{d}[Nina] sanger d
\role{e}[Isolde] Student

\end{roles}

\begin{props}
    \prop{Tøj} []Alle fire sangere har forskelligt farvet tøj på
\prop{Kittel og handsker}

\end{props}


\begin{sketch}

\scene{Klinisk, koldt, - spacy, lidt varmere når sangerne begynder at harmonere}
\scene{Sort lys ned}

\scene{Første sanger kommer ind på scenen. Begynder at varme stemmen op.}
\says{sanger a} “mi mi miiii”

\scene{Efter et øjeblik kommer den næste sanger ind på scenen, klædt i en anden farve. Stiller sig over i den anden ende og begynder også at varme op.}
\says{sanger b} “la la laaai”

\scene{Tredje og fjerde sanger kommer ind i hver side af scenen, i hver deres farver også. Begynder at varme op sammen med de andre sangere.}

\says{sanger c} “uh uh uh”
\says{sanger d} “å å å ååi”

\scene{lyset ændres så det er tydeligt at der nu starter noget nyt}

\scene{En person kommer ind, med kittel på. Tager handsker på og gnider (spændt) hænderne sammen, og begynder at guide sangerne rundt på scenen, indtil de er blevet “sorterede” og står side om side med hinanden. Når sangerne er blevet sorterede begynder de at harmonisere.}

\scene{Efter 5 sekunders harmonisering stopper sangerne med at synge igen.}

\says{Student} \act{klapper og siger:} Åh, det var dejligt, så fik jeg lige ordnet den sanger-sekventering.


\scene{Sort lys op.}

\end{sketch}
\end{document}
