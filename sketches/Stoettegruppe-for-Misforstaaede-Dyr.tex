\documentclass[a4paper,11pt]{article}

\usepackage{revy}
\usepackage[utf8]{inputenc}
\usepackage[T1]{fontenc}
\usepackage[danish]{babel}

\revyname{Biorevy}
\revyyear{2022}
% HUSK AT OPDATERE VERSIONSNUMMER
\version{0.1}
\eta{$n$ minutter}
\status{Cringe}
\responsible{En revyst}

\title{Støttegruppe for Misforståede Dyr}
\author{Caroline, Jonathan, Emma og Tristan}

\begin{document}
\maketitle

\begin{roles}
    \role{SO}[Amanda] Stålorm
 \role{T}[Nethe] Tæge
 \role{F}[Claudia] Flåt
 \role{O}[Sofie] Sofie
 \role{M}[Andrea] Mejer Maya
 \role{S}[Oscar] Svirreflue
 \role{Mark}[Laura] Firben
 \role{K}[-] Kamel
\end{roles}

\begin{props}
    \prop{Puder}[Andrea skaffer puder]
\end{props}

\begin{sketch}

\scene{Sort lys ned}

\scene{5 folk i dyrekostumer sidder i en halvcirkel med en person, helst iført tweet, sidder med dem(ordstyren). Ordstyren sidder i midten, tæge og flåt sidder på hver sin side af ordstyren. Ved siden af tæge sidder mejer og svirreflue. Ved siden af flåt sidder stålorm og ved siden af stålorm er der en tom stol}

\says{o} hej alle sammen, og velkommen til denne uges session, her i de misforståede dyrs støttegruppe. Husk på at dette er en safe zone, så tag det roligt, her er vi alle sikre. Nå hvem vil være den første til at snakke i dag, hvad med dig flåt?

\says{t} JEG ER ALTSÅ EN TÆGE! Flåt han sidder jo lige dér! Vi ligner jo ikke engang hinanden! Han er ikke engang et insekt! Jeg har 6 ben! Der er SÅ mange forskellige tæger, og hvorfor er det at jeg skal associeres med en blodsugende parasit?!

\says{o} Ej flå, nej jeg mener tæge. Det var altså ikke særlig pænt at kalde andre for parasitter 


\says{f}\acts{flåt er meget arrogant og slet ikke forstående} Det er da okay. Det rør ikke mig, jeg er jo en parasit. Jeg kan jo ikke gøre for, at alle lægger mærke til mig, og i bliver nødt til at leve i skyggen af mig for at få opmærksomhed
\says{t} Jamen i det mindste kan folk huske mit navn! 
\acts{f, t} begynder at skændes hen over ordstyren  
\says{o} Så i to! Ej nu stopper i altså! SÅ LUKKER VI FOR TÆGERNE!
\acts{mejer er meget optimistisk}
\says{m} Mit navn er Maja og alle tror jo jeg er en edderkop. Og det er sådan set også meget fint. Altså en edderkop har 8 ben og… Ja det har jeg sådan set også. Ja jeg kan jo ikke lave spind, men det er hvad det er. Nogengange bliver vi også forvekslet, når vi sidder ude i brusekabinen, ja, mange af mine brødre er blevet slået ihjel på grusomme måder… Men det må man jo tage med.
\says{o} Ja, men nu skal alt ikke handle om dig, det var jo faktisk stålorms tur. Klamme edderkop. 
\acts{stålorm er meget genert og vemodig. Stålorm kan ikke leve op til forventningerne om at han/hun er en slange}
\says{s} Jamen… Jeg ved jo ikke om jeg har noget at sige. 

\says{o} Kom nu stålorm, fortæl hvordan det er at være en slange. 

\says{s} Ja! Det er fantastisk at være en slange! Man er giftig, man kan kvæle ting og man spiser mus! Men… Når nej, det kan jeg slet ikke. Jeg er jo ikke en slange. Jeg er ikke engang en orm! Når folk ser mig ude i skovene siger de altid “Se en slange”, men så bliver jeg altid så forskrækket at jeg taber halen. Og så siger folk altid “Ej hvor kedeligt, det er jo bare et firben” 

\says{o} Men i det mindste har du ikke tabt modet! Nå hvem skal nu åbne op?

\says{m} der var engang en gruppe børn der fangede mig og pillede alle mine ben af. 

\says{o} Ja Maja, nu gider vi altså ikke høre mere på dig. Hvad har svirreflue at sige? 

\says{s} Ej! Jeg hedder HVEPS! Eller til nøds kan jeg hedder svirrehvips. Eller mine rigtig gode venner kalder mig GEDEHAMS! Ja og dumme mennesker kalder mig for bi! 

\says{o} Vi har snakket om det her, du er en flue

\says{s} Nej nej! Se nu på min farve og jeg har også den bråd!

\scene{firben kommer ind på scenen}
\says{mark} Undskyld jeg kommer for sent, jeg tabte min hale på vejen.

\says{o} Ah! Mark Zuckerberg. Godt du kunne komme idag! 

\scene{ALTERNATIV SLUTNING}

\scene{kamel kommer ind på scenen}
\says{k} Undskyld jeg kommer for sent, jeg farede vild i zoologisk have…

\says{o} Ah! dromedar! Godt du kunne komme idag!
\acts{kamel græder/ser meget opgivende ud}


\scene{Sort lys op.}

\end{sketch}
\end{document}
