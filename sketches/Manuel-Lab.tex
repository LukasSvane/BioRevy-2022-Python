\documentclass[a4paper,11pt]{article}

\usepackage{revy}
\usepackage[utf8]{inputenc}
\usepackage[T1]{fontenc}
\usepackage[danish]{babel}

\revyname{Biorevy}
\revyyear{2022}
% HUSK AT OPDATERE VERSIONSNUMMER
\version{0.1}
\eta{$n$ minutter}
\status{}
\responsible{En revyst}

\title{Manuel Lab}
\author{Lukas}

\begin{document}
\maketitle

\begin{roles}
   \role{A}[Line] DJØF'er
\role{B}[Søren R] Forsker
\role{C}[Isolde] Forsker

\end{roles}

\begin{props}
   \prop{1 kontorstol}[find et sted på DIKU]
\prop{1 pult}
\prop{1 glas vand}[evt. DIKU kantinen]
\prop{4 sæber (af hvid chokolade)}[Det skal omsmeltes en gang]

\end{props}

\textbf{Bemærkninger til Teknikken:} Grønt lys

\begin{sketch}
\scene{Sort lys ned}

\scene{Uddannelsesminister/eller en anden kommer ind på scenen}

\says{A} Hej allesammen. Jeg kommer fra KU’s administration, og jeg vil gerne præsenterer jer for hvordan man kan få forskningspengene til at række meget længere, hvis bare man laver et par små ændringer. Og så kommer vi også til at være meget mere bæredygtige.

\scene{B og C kommer ind sammen med en kontorstol.}

\says{A} Her er vores første forslag. Det er helt unødvendigt at bruge en centrifuge. Man kan nemlig sagtens centrifugere på en meget nemmere måde.

\scene{C sætter sig i kontorstolen og strækker armene ud i begge retninger. B giver dem en falcon tube i hver hånd, og spinner dem rundt på stolen. C bliver meget rundtosset og B hjælper C ud af scenen.}

\says{A} Og det er også helt slut med at bruge vortex nu. Du skal bare brygge kaffen ekstra stærk. (Eller… Nu hvor vi har slukket for varmen går alle alligevel rundt og fryser…)

\scene{En person kommer ind på scenen (a) med kaffekop i hånden og store rysteture. Den får stukket et rør i den anden hånd af person (b), som de så ryster godt.}

\scene{Et bord står klar med en citron, et glas vand og en sæbe (af hvid chololade)}

\says{A} Desuden har vi også fundet et smart og nemt alternativ til at bruge pH-meter. Det kan du nemlig sagtens gøre helt selv. Først skal du lige kalibrere. Det kan du gøre ved at spise noget surt og noget basisk. Jeg har faktisk et stykke citron og et stykke sæbe til dig, lige her!

\scene{C sker skeptisk ud men tager en bid af hver.}

\says{A} Og så kan du bare selv smage dit præparat.

\scene{A rækker C en falcon tube, som C drikker. C falder død om.}

\says{A} Uha, den var vist lidt sur.

\scene{Sort lys op.}

\end{sketch}
\end{document}
