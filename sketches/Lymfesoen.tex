\documentclass[a4paper,11pt]{article}

\usepackage{revy}
\usepackage[utf8]{inputenc}
\usepackage[T1]{fontenc}
\usepackage[danish]{babel}

\revyname{Biorevy}
\revyyear{2022}
% HUSK AT OPDATERE VERSIONSNUMMER
\version{0.1}
\eta{$3$ minutter}
\status{?}
\responsible{En revyst}

\title{Lymfesøen}
\author{Lukas, Amanda og Tristan}

\begin{document}
\maketitle

\begin{roles}
   \role{C}[Claudia] Inficeret celle
\role{V}[Sofie] Virus 
\role{T}[Andrea] T-celle

\end{roles}

\begin{props}
    \prop{DØDSLIGANDEN}[Person, der skaffer] 
    \prop{Forskellige våben, sværd, pistol, etc.}[Person, der skaffer]
    \prop{Kuffert}[Person, der skaffer]
\end{props}

\textbf{Bemærkninger til Teknikken:} Svanesøen af Tjajkovskij skal sættes på, spørg de medvirkende hvilken version og så videre de vil have

\begin{sketch}
\scene{Sort lys ned}

\scene{0:00 min Svanesøen spiller, Celle går rundt og har det godt}
\says{V} \scene{0:27 min. Virus kommer ind på scenen, danser sig nærmere på cellen, danser om cellen, forførende}
\says{V} \scene{0:56 min. Virus inficerer den raske celle}
\says{C} \scene{0:56 min. celle bliver syg (åh nej)}
\says{T} \scene{1:19 min. T-cellen enter stage right, heroisk. T-celle spiller med musklerne. T-celle leder efter syg celle}
\says{C} \scene{1:44 min. Den syge celle bliver fundet, begynder episk jagtscene}
\says{T} \scene{1:53 min. Mere dramatisk, syg celle vælter over kuffert og trygler for sit liv}
\says{T} \scene{2:03 min. T-celle finder våben frem, derefter det perfekte våben (DØDSLIGANDEN) og viser frem}
\says{T} \scene{2:25 min. Første stik med DØDSLIGANDEN, syg celle stikkes i takt til de dramatisk slag i musikken. Måske har syg celle også en dødsreceptor (Fas), som T-cellen finder frem}
\says{C} \scene{2:36 min. Syg celle dør. Måske smider den om sig med organeller}
\says{T} \scene{2.42 min. T-celle går stolt ud}

\scene{Sort lys op.}
\end{sketch}

\end{document}