\documentclass[a4paper,11pt]{article}

\usepackage{revy}
\usepackage[utf8]{inputenc}
\usepackage[T1]{fontenc}
\usepackage[danish]{babel}

\revyname{Biorevy}
\revyyear{2022}
% HUSK AT OPDATERE VERSIONSNUMMER
\version{0.1}
\eta{$uendelige$ minutter}
\status{}
\responsible{Ania}

\title{Specialehelvedet}
\author{Søren, Laurids, Anna, Nicholas, Tromle, Emma}

\begin{document}
\maketitle

\begin{roles}
    \role{F}[Ania] Fortæller
\role{S}[Line] Studerende
\end{roles}

\begin{props}
    \prop{Rekvisit}[Person, der skaffer] Rekvisitforklaring
\end{props}

\textbf{Bemærkninger til Teknikken:} Når det brænder: Ildlys og brændelyd/knitrende træ og røg \\Lastbil og crashlyd til slut :)

\vspace{0.3cm}

\textbf{Ideer:}
%% Disse linjer er ikke gyldige
% \\Bord + stol + Falsk computer + kaffekop  
% \\Telefon + oplader  
% \\Cykel  
% \\Fortælleren vælter kaffekoppen meget casual 
% \\Speciale studerende
% \\Hverdagstøj evt. joggintøj (hættetrøje ish) 
% \\Grå jogginbukser med kaffeplet i skødet
% \\Briller 
% \\Bord i modsatte side af sceneåbning (evt ind af bagtæppet) 
% \\Blåt lys på bord hvis muligt
% \\Skiftende spot mellem fortæller og studerende evt (spørg teknikken) 


\begin{sketch}

\scene{En studerende sidder i sin lejlighed med sin computer og skriver speciale. Den studerende er glad og optimistisk}
\scene{Sort lys ned}

\scene{ Specialestuderende sidder i sin lejlighed og skriver speciale. } 

\says{F} Engang for et par dage siden, sad en klog og flittig specialestuderende, og var ved at lægge sidste hånd på sin specialeafhandling. Hun/han havde arbejdet i 100 dage og 100 nætter utrætteligt. Hun var Micheal Angelo og opgaven var hendes/hans sixstinske kapel. Dette var hendes/hans mesterværk. 


\says{F} Hun havde lige skænket sin sidste kop kaffe, og glad og fornøjet, en optimistisk pige som ikke lod sig slå ud af noget. Nu var hun klar til at aflevere sit speciale.

\says{S} En dag er hvad man gør den til. Med god attitude og gåpåmod, kan intet slå mit humør ud af kurs.

\says{F} Men hvad hun ikke vidste var, at dagen ville tage en hel anden drejning. 

\says{F} Bedst som hun skulle til at skrive titlen på sin specialeafhandling, begyndte “S” tasten at holde op med at virke. Dette gjorde det svært at skrive titlen “Sanger Sekventering af Resistente Staphylococcus aureus stammer” 

\scene{Titlen “Sanger Sekventering af Resistente Staphylococcus aureus stammer”  skal være på AV}

\says{S} Nej, nej, nej! FUCK! 

\says{F} Og dette ændrede hendes titel til noget helt andet.

\scene{Titlen ændres til “Anger Ekventering af Reitente Taphylococcu Aureu Tammer” på AV}

\says{S} Nå, men der skal da mere end en ødelagt S-tast til at slå mig ud

\says{F} Men ikke nok med dét spildte hun også sin nyophældte kaffe ud over sig selv

\says{S} Av for satan.. Aaaah 

\says{F} men nu skulle det for alvor tage fart. Pludselig holdt hendes internet op med at virke. 

\says{S} Jamen mit internet virker da? 

\says{F} En SMS fra YouSee tikker ind, som fortæller at hendes internet skal opdateres, og hun har intet WIFI før i morgen.

\says{S} Ååååårhh. Hmmmm. Jeg laver et hotspot fra min telefon! 

\says{F} men hendes telefon var gået ud

\says{S} Arh pis! Jeg henter min oplader

\says{F} Desværre gik hendes oplader, meget spontant, i brand. 

\says{S} Ej please! Kan den ikke lade være med det? 

\says{F} Okay… Meget spontant brød hendes lejlighed i brand

\scene{Der kommer rødt lys og røg på scenen, lejligheden går i brand}

\acts{Hoster, tager computeren og løber ud} 

\scene{cykel kommer på scenen} 

\acts{kommer hen til sin cykel}

\says{F} Nu var hun tvunget til at cykle hen til universitetet, men til hendes store forbavselse var hendes cykel blevet piftet 

\says{S} Det er da ikke piftet? 

\says{F} *I bestemt tone* Desværre var hendes cykel piftet.

\says{S} Nej. Den. Er. Ej. 

\says{F} Dagen kunne altid blive værre

\says{S} Årh mand… 

\acts{S} Sætter sig på hug og lukker luft ud af sin egen cykel

\says{F} Nu blev hun desværre nødt til at komme hen til campus til bens. Tiden var ved at 
rinde ud. Kun en time til aflevering. 

\acts{S} begynder at gå ud af scenen

\says{F} Hvad hun desværre ikke lagde mærke til, var lastbilen der kom fra højre side, og brækkede begge hendes arme. Desværre nåede hun ikke at aflevere. 

\scene{en lastbil lyd kommer fra AV}

\says{S} Avvvvv….

\says{F} Hvad kan vi lære af dette? Din dag kan altid blive værre, medmindre du er Olivia. 

\scene{Sort lys op.}

\end{sketch}
\end{document}
