\documentclass[a4paper,11pt]{article}

\usepackage{revy}
\usepackage[utf8]{inputenc}
\usepackage[T1]{fontenc}
\usepackage[danish]{babel}

\revyname{Biorevy}
\revyyear{2022}
% HUSK AT OPDATERE VERSIONSNUMMER
\version{0.1}
\eta{$2$ minutter}
\status{Noget}
\responsible{Freja}

\title{Russere i Zoo}
\author{Nina og Tristan}

\begin{document}
\maketitle

\begin{roles}
\role{V}[Line] Vejleder
    \role{R1}[Søren R] Rus
\role{R2}[Freja] Rus 2
\role{R3}[Ania] Rus3
\end{roles}

\begin{props}
    \prop{1 bord}[Person, der skaffer]
    \prop{Kattebamse}[Skaffer selv]
\end{props}


\begin{sketch}

\scene{Beskrivelse}
\scene{Sort lys ned}
\scene{V, R1, R2 og R3 kommer ind. V er træt}
\says{R1} Ej hvor er det spændende at komme ud i felten.
\says{R2} Jeg elsker også bare at være i zoo
\says{R3} ja, der er bare så mange søde dyr.
\says{R1} Ja som den der øhh MYRESLUGER
\says{V} Nej, nu er det altså en tapir det der, men de minder også lidt om hinanden.
\says{R2,R3} (i kor) nårhhh
\says{V} Ja man lærer noget nyt hver dag
\says{R3} Hvad med den der?
\says{R2} Den kender jeg! Den hedder STRUDS!
\says{V} arhh nu er det ikke HEELT rigtigt, det er godt nok det man kalder en emu. 
\says{R1} En fugl er vel en fugl ik?
\says{V} De har i hvert fald mange fællestræk…
\says{R2} Se dem her! de er vildt nusser.
\says{R3} aj ja bævere er bare så nuttede.
\says{V}  \act{sukker, ikke som glukose men du ved når man er lidt træt af det hele} Nu er lige præcis dem der jo så oddere… Det står faktisk også derhenne på skiltet
\says{R1,R2,R3} Nårhh
\says{R2} Jeg har engang set en bæver og den lignede altså den der!
\says{V} Så har det måske været en odder du så…
\says{R2} Nårhh det havde jeg ik’ tænkt på…
\says{R1} Arhh LØB tigeren er sluppet ud!
\says{V} HVAD HVOR!
\says{R2,R3} DER! ARHHH
\says{V}  \act{sukker dybt, igen ikke glucose der er faldet langt ned stadig bare træt af det hele} Nu stopper i altså, har i overhovedet læst kompendiet! Det der er bare en helt normal huskat.
\says{R1} Det vidste jeg da godt, jeg skulle teste jer.
\says{R2,R3}\act{griner}
\says{R3} Ej hvor er du bare sjov, tihi.
\acts{R2 peger}
\says{R2} De er så sjove dem der, dem har jeg set i cirkus! 
\says{R3} Er det ikke den der hedder en dromedar!
\says{V} Nej nu er det simpelthen nok!
\scene{går over i “ingen dromedar” sang}


\end{sketch}
\end{document}
