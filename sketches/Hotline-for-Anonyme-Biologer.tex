\documentclass[a4paper,11pt]{article}

\usepackage{revy}
\usepackage[utf8]{inputenc}
\usepackage[T1]{fontenc}
\usepackage[danish]{babel}

\revyname{Biorevy}
\revyyear{2022}
% HUSK AT OPDATERE VERSIONSNUMMER
\version{0.1}
\eta{$n$ minutter}
\status{}
\responsible{En revyst}

\title{Hotline for Anynyme Biologer}
\author{Freja, Sofie og Lukas}

\begin{document}
\maketitle

\begin{roles}
\role{a} [Oscar]Hotline besvarer
\role{b} [Nethe] Biolog 1 
\role{n} [Astrid] Biolog 2 

\end{roles}

\begin{props}
    \prop{Rekvisit}[Person, der skaffer] Rekvisitforklaring
\end{props}


\begin{sketch}

\scene{Beskrivelse}
\scene{Sort lys ned}

scene{På scenen sidder hotline-besvareren og strikker eller noget andet lignende. Pludselig kommer der en lyd af en telefon, der ringer, som hotline-besvareren tager.}

\says{HB} Goddag, de har ringet til vores hotline - hvad kan jeg hjælpe med? Vi minder om, at alle opkald selvfølgelig er anonyme og at vi har tavshedspligt.

\says{B1} Hej, øh… altså… det er bare fordi jeg er sådan lidt nervøs. Ser du… det er nemlig fordi det er min første tur i gummi/med gummi på, og jeg har bare en masse spørgsmål.

\says{HB} Vi er glade for at du ringer. Det er helt normalt at være usikker første gang. Men du kan bare spørge os, så skal vi nok give dig nogle råd.

\says{B1} Altså… Først og fremmest så er jeg bare lidt i tvivl - hvordan finder man den rigtige størrelse?

\says{HB} Ja, det kan jo godt være lidt svært lige at finde den rigtige størrelse. Men her er det vigtigt at være opmærksom på, hvordan det sidder. Det er vigtigt, at pasformen er god. Det må ikke være for løst eller for meget plads. Men det må heller ikke være for stramt, og hvis det er så stramt at du helt begynder at miste følelsen, så ved du i hvert fald den er gal.

\says{HB} Men altså, det er vigtigt at størrelsen er den rigtige, så det er ikke lige på det her tidspunkt, du skal pynte på den.

\says{B1} Nå ja, det kan jeg da godt se… Men hvad nu hvis man for eksempel har fået nogle der er for små, og man ikke kan få dem af igen?

\says{HB} Ja, det kan jo ske. Men så kan man jo hjælpes ad sammen med at komme ud. Så kan man også bruge det som en mulighed for at lære hinanden bedre at kende.

\says{B1} Ja og så er det også fordi jeg sådan har lidt sart hud, og jeg er lidt bange for at man måske kan få udslæt for eksempel.

\says{HB} Det er meget få der er allergiske, men hvis du tidligere har haft mistanke om latex-allergi, så bør du lige snakke med din læge først. Og ellers så kan de også fås i neopren.

\says{B1} Okay, det er jeg rigtig glad for at høre. Så tænkte jeg, altså bare sådan rent hypotetisk… kan de også tåle bid?

\says{HB}

\says{B1} Hvad sker der hvis der går hul?

\says{HB} Du kan jo prøve at spørge en ven, om han vil hjælpe dig med din første gang

\says{B1} Jo, det vil jeg prøve 
\scene{Han ringer til en ven, som kommer ind på scenen og de går ud bag bagtæppet. En masse stønnen og massen. Endelig kommer han ud igen, iført waders.} 

\says{HB}

\says{a} De passer jo perfekt! Det var slet ikke så svært at få dem på. Nu føler jeg mig godt beskyttet og klar til at komme afsted.

\scene{Sort lys op.}

\vspace{1cm}

\textbf{Q\&A:}

\says{Q} Hvad nu hvis der kommer hul i den? 
\says{A} hvis der er hul i skal den lappes først. Ellers så skal man ikke finde en anden. 

\says{Q} Kan man få udslæt, når den rører huden?
\says{A} Nej, der er meget få der er allergiske, men hvis du tidligere har haft mistanke om latex-allergi, så bør du lige snakke med din læge først

\says{Q}Skal den vaskes efter brug?
\says{A}Ja, det går jo ikke at det ligger og bliver dårligt eller begynder at lugte. 

\says{Q}Kan den genbruges? 
\says{A}De kan jo være lidt dyre, så hvis der ikke er hul i det og man vasker det godt og så kan den godt bruges igen.

\says{Q}Hvor længe kan de holde? kan den blive for gammel?
\says{A}Ja, de kan godt blive for gamle. Når de bliver for gamle kan de godt miste deres elasticitet, og så er der større risiko for lækage.  

\says{Q}Hvad hvis man ikke kan få den af (eller sidder fast)?
\says{A}Så må man hjælpes ad med at få den af. Så kan man måske også tage muligheden for at lære hinanden bedre at kende.

\says{Q}Hvad nu hvis man ikke kan få den på, eller den sidder for løst?
\says{A}  

\says{Q}Hvordan ved man at det er den rigtige størrelse?
\says{A}Den skal hverken være for lille, eller for stor. Den skal slutte tæt, men den må heller sidde så tæt, så det er ubehageligt, eller stopper blodtilførslen. Men det er vigtigt at have nogen, i den rigtige størrelse, så det er ikke her, du skal være pinllig over din størrelse

\says{Q}Hvor lang tid må man have dem på?
\says{A}Hvis man har dem på i længere tid, så kan

\says{Q}Hvad nu hvis den bliver fyldt op?
\says{A}Det er ikke så godt, så beskytter den dig ikke så godt længere

\says{Q}Men kan man så stadig mærke noget igennem den?
\says{A}Ja altså det gør at man ikke mærker omgivelserne helt så tydeligt, men det er bestemt stadig den samme oplevelse omend lidt sikrerer. 

\says{Q}Hvad nu hvis man bliver bidt mens man har det på?
\says{A}Tja, det beskytter vel også mod bid, det giver da i hvert fald et ekstra lag man skal igennem

\says{Q}(nervøst) Hvad nu, hvad nu hvis jeg kommer ud hvor jeg ikke kan bunde?
\says{A}Det er altid helt okay at trække i land igen. Det kan være at den eller de andre bliver lidt skuffede, men det er altid bedst 

\end{sketch}
\end{document}
